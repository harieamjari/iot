\documentclass[12pt, a4paper]{article}
\usepackage{graphicx}
\usepackage{tikz}
\usetikzlibrary{shapes.geometric, arrows, positioning}
\usepackage{hyperref}
\usepackage{geometry}

\geometry{
 a4paper,
 total={170mm,257mm},
 left=20mm,
 top=20mm,
}

\title{\textbf{System Architecture for IoT Project}}
\author{Draft for harieamjari/iot}
\date{\today}

\begin{document}

\maketitle

\begin{abstract}
This document outlines the system architecture for the Internet of Things (IoT) project. The system is designed to facilitate real-time data monitoring and control between physical sensing nodes and a centralized cloud interface. The architecture is divided into three primary layers: the Perception Layer (Edge), the Network Layer, and the Application Layer.
\end{abstract}

\section{System Overview}
The proposed system integrates embedded microcontrollers with environmental sensors to collect data. This data is transmitted wirelessly to a central server where it is processed, stored, and visualized for the end-user.

\section{Architectural Layers}

\subsection{1. Sensing layer}
At the lowest level, the system measure voltage across different sensors,
which in turns, turns into data. The data collected are from 2 LDRs (Light
Dependent Resistors), and from the level of battery is fed into a voltage
divider.

\begin{itemize}
	\item \textbf{LDR 5549 (45K - 140K Ohm)}: An LDR is connected in series with 100k ohm
resistor. the other end of the LDR is connected to 3.3V while the other end of the resistor
is connected to GND, forming a voltage divider that takes in very little r current. Voltage
level is read in the middle.
    \item \textbf{Battery voltage detection}: Same as before, two 100K ohm is connected in series and voltage in between is read.
\end{itemize}

\subsection{2. Network Layer}
This layer handles the transmission of data between the edge devices and the cloud server.
\begin{itemize}
    \item \textbf{Gateway}: The ESP 32 built in Wi-Fi module opens up a WiFi access point
for users to connect to and control. The MCU, could also be program to connect to already existing
WiFi access point of creating a new one to allow users flexibility in controlling the MCU
		in their home WiFi.
 
\end{itemize}

\subsection{3. Data processing layer}
This layers handles the processing of data collected from the sensors. 

The ESP32 currently does all of the processing of the data collected. It has
been programmed in C manually to perform the task of switching leds and
reacting to HTTP POST request from users.
\subsection{4. Application layer}
This layer is what is actually exposed to the user.

The ESP32 exposes a very simple API that allows users to see which lights are
actives. It also allows users to turn on/off light over the web interface and 
even add a schedule to automatically turn on/off a light at specific time.

\section{System Diagram}

%\begin{figure}[h]
%\centering
%\begin{tikzpicture}[node distance=2cm]
%
%% Nodes
%\node (sensor) [rectangle, rounded corners, minimum width=3cm, minimum height=1cm, text centered, draw=black, fill=blue!10] {Sensors \& Actuators};
%\node (mcu) [rectangle, rounded corners, minimum width=3cm, minimum height=1cm, text centered, draw=black, fill=green!10, below of=sensor] {Microcontroller (ESP32/8266)};
%\node (network) [cloud, draw,blue, fill=white, aspect=2, below of=mcu] {Internet / MQTT Broker};
%\node (server) [rectangle, rounded corners, minimum width=3cm, minimum height=1cm, text centered, draw=black, fill=orange!10, below of=network] {Cloud Server \& Database};
%\node (user) [rectangle, rounded corners, minimum width=3cm, minimum height=1cm, text centered, draw=black, fill=red!10, below of=server] {User Dashboard (Web/App)};
%
%% Arrows
%\draw [thick, ->] (sensor) -- (mcu);
%\draw [thick, <->] (mcu) -- node[anchor=west] {Wi-Fi / MQTT} (network);
%\draw [thick, <->] (network) -- (server);
%\draw [thick, <->] (server) -- (user);
%
%\end{tikzpicture}
%\caption{High-Level IoT System Architecture}
%\label{fig:arch}
%\end{figure}
%
%\section{Conclusion}
%The described architecture ensures scalability and low-latency communication. By decoupling the sensing logic from the data visualization, the system allows for easy addition of new sensor nodes without significant reconfiguration of the backend infrastructure.
%
\end{document}
